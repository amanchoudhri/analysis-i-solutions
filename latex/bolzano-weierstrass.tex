\documentclass[11pt, oneside]{article}
\usepackage[margin=1in]{geometry}
\geometry{letterpaper}
\usepackage{graphicx}

% \setlength\parindent{0pt}

\usepackage{mathtools}
\usepackage{enumitem}

\usepackage{amsmath}
\usepackage{amssymb}
\usepackage{amsthm}

\newcommand{\R}{\mathbb{R}}
\newcommand{\N}{\mathbb{N}}

\makeatletter
\renewcommand*\env@matrix[1][*\c@MaxMatrixCols c]{%
  \hskip -\arraycolsep
  \let\@ifnextchar\new@ifnextchar
  \array{#1}}
\makeatother

\newlist{alphalist}{enumerate}{1}
% label must be specified using \setlist
\setlist[alphalist,1]{label=\alph*.}

\newtheorem*{theorem}{Theorem}
\newtheorem{lemma}{Lemma}
\newtheorem{prop}{Proposition}

\theoremstyle{definition}
\newtheorem*{definition}{Definition}

\title{Subsequences and the Bolzano-Weierstrass Theorem}
\author{Aman Choudhri - \texttt{aman.choudhri@columbia.edu}}

\begin{document}

\maketitle
I found Terrence Tao's treatment of subsequences in his \emph{Analysis I} to be fascinating, so I decided to typeset and publish a small set of notes on the section.

\begin{definition}
    Given a sequence $(a_n)_{n=0}^\infty$, say $(b_n)_{n=0}^\infty$ is a subsequence iff there exists a
    strictly increasing function $f: \mathbb{N} \rightarrow \mathbb{N}$ such that
    $$b_n = a_{f(n)} \text{ \ for all $n \in \mathbb{N}$.}$$
\end{definition}

For example, let $(a_n)_{n=0}^\infty$ be the sequence
$$0, 1, 2, 3, ...$$
Then the sequence containing only evens
$$0, 2, 4, 6, ...$$
is a subsequence, as is the sequence containing only primes
$$2, 3, 5, 7, ....$$

Unpacking the definition a bit, I like to think of $f$ as sort of a ``selection'' function,
which picks elements from $(a_n)_{n=0}^\infty$ to put into a new sequence. In the above example of the sequence
containing only the evens, the function $f: \mathbb{N} \rightarrow \mathbb{N}$ would be defined by
$f(n) = 2n$.

\begin{prop}
    Let $(a_n)_{n=0}^\infty$ be a sequence of real numbers, and let $L \in \R$. Then the following statements are equivalent:
    \begin{enumerate}[label=\emph{(\alph*)}]
        \item The sequence $(a_n)_{n=0}^\infty$ converges to $L$.
        \item Every subsequence of $(a_n)_{n=0}^\infty$ converges to $L$.
    \end{enumerate}
\end{prop}

\begin{proof}
    First, prove that (b) implies (a).
    Assume every subsequence of $(a_n)_{n=0}^\infty$ converges to $L$.
    Notice that $(a_n)_{n=0}^\infty$ is a subsequence of itself, given by the function $f(n) = n$, and so it converges to $L$ by assumption.
    
    Next, prove that (a) implies (b). 
    Take some subsequence $(b_n)_{n=0}^\infty$ given by $b_n = a_{f(n)}$ for some strictly increasing function $f$.
    To show that $(b_n)_{n=0}^\infty$ converges to $L$, let $\epsilon > 0$ and find some integer $N$ such that for all $n \geq N$ we have $|b_n - L| < \epsilon$.
    First, note that since $f$ is strictly increasing, we have $f(n) \geq n$ for all $n$. This is easily shown by induction:
    $f(0)$ is a natural number, so we must have $f(0) \geq 0$.
    And assuming $f(n) \geq n$, $f$ strictly increasing gives us $f(n + 1) > f(n) \geq n$ so $f(n + 1) \geq n + 1$.

    With this, use convergence of $(a_n)_{n=0}^\infty$ to get an integer $N$ such that for all $n \geq N$, we have
    $| a_n - L | < \epsilon$.
    Since $f(n) \geq f(N) \geq N$ for all $n \geq N$, we have in particular that $| a_{f(n)} - L | < \epsilon$. But $a_{f(n)} = b_n$, so
    $$| b_n - L| < \epsilon,$$
    completing the proof. 
\end{proof}

The first direction was relatively trivial, but I think the second highlights
the importance of defining our ``selection'' function $f$ to be strictly increasing. If it was
only restricted to be nondecreasing, for instance, we could let $f(n) = 0$ for all $n$. Then 
$$1, 1, 1, ...,$$
which clearly converges to $1$, would be a subsequence of
$$1, \frac{1}{2}, \frac{1}{3}, ...,$$
which converges to $0$. It would also be a valid subsequence of
$$1, 2, 3, ...,$$
which doesn't converge at all! Thus the strict increasing requirement corresponds to the idea of
``filtering out'' elements from the original sequence to create a new sequence, without
adding or duplicating anything.


\begin{prop}\label{subseqconvtolimpt}
    Again let $(a_n)_{n=0}^\infty$ be a sequence of real numbers, and let $L \in \R$. Then the
    following statements are equivalent:
    \begin{enumerate}[label=\emph{(\alph*)}]
        \item $L$ is a limit point of $(a_n)_{n=0}^\infty$.
        \item There exists a subsequence of $(a_n)_{n=0}^\infty$ which converges to $L$.
    \end{enumerate}
\end{prop}

\begin{proof}
    First show that (b) implies (a). Let $(b_n)_{n=0}^\infty$ be a subsequence that converges to $L$, given by $b_n = a_{f(n)}$.
    Take some $\epsilon > 0$ and some natural number $N$. To show that $L$ is a limit point of $(a_n)_{n=0}^\infty$, we want some $n \geq N$ such that $|a_n - L| < \epsilon$.
    Since  $(b_n)_{n=0}^\infty$ converges to $L$, we can take some $M$ such that for all $m \geq M$,
    $$|b_m - L| < \epsilon$$
    Rewriting, we have that 
    $|a_{f(m)} - L| < \epsilon$
    for all $m \geq M$. Now let $n := \operatorname{max}(f(N), f(M))$. Since $f$ is strictly increasing,
    $$n \geq f(N) \geq N \quad \text{and} \quad n \geq f(M) \geq M$$
    Thus we have some $n \geq N$ such that $|a_n - L| < \epsilon$, and we're done.

    \vspace*{0.5em}
    Next show that (a) implies (b). We want
    to define a sequence of natural numbers
    $m_0, m_1, ...$ such that the sequence given by $b_n := a_{m_n}$ converges to $L$.
    Define our first term by
    $$m_0 := \min \{ k \in \N : |a_k - L| < 1 \}$$
    The above set is clearly bounded below by $0$. To show that it is nonempty, take $\epsilon = 1$ and $N = 0$ in the definition of a limit point to get some $k \in \N$ with $|a_k - L | < 1$. Thus the minimum exists and $m_0$ is well defined.
    Next, recursively define the rest of the sequence as follows:
    $$m_{n+1} := \min \left\{ k > m_n : |a_k - L| < \frac{1}{n + 1} \right\}$$
    Again, the set is clearly bounded below by $m_n$. Similarly to the argument above, the set is nonempty since we can take
    $\epsilon = \frac{1}{n+1}$ and $N = m_n$ in the definition of a limit point to find some $k \geq m_n$ with $|a_k - L| < \frac{1}{n+1}$. Thus the minimum $m_{n+1}$ exists and is well-defined.

    Note that $(m_n)_{n=0}^\infty$ is strictly increasing by construction, so the sequence $(b_n)_{n=0}^\infty$ given by $b_n = a_{m_n}$ is a subsequence.
    Now show that $(b_n)_{n=0}^\infty$ converges to $L$. Take some $\epsilon > 0$, and let $N$ be some natural number such that $N > \frac{1}{\epsilon}$.
    Using our definition of $(b_n)_{n=0}^\infty$, we see that for all $n \geq N$ we have
    $$|b_n - L| = |a_{m_n} - L| < \frac{1}{n + 1} < \frac{1}{N} < \epsilon,$$
    and thus $(b_n)_{n=0}^\infty$ converges to $L$.

\end{proof}

With these facts established, we now turn to the Bolzano-Weierstrass theorem, which states that every bounded sequence has a convergent subsequence.
Before we prove the theorem, though, it will be useful to establish the following lemmas.

\begin{lemma}\label{limpts}
    Let $(a_n)_{n=0}^\infty$ be a sequence with $\limsup_{n \rightarrow \infty} a_n = L$ for some $L \in \R$.
    Then $L$ is a limit point of $(a_n)_{n=0}^\infty$.
\end{lemma}

\begin{proof}
    Take $\epsilon > 0$ and some natural number $N$. The limit superior is defined to be
    $$L := \inf (a_n^+)_{n=0}^\infty,$$
    where $a_k^+ := \sup (a_n)_{n=k}^\infty$. Firstly, notice that the sequence $(a_n^+)_{n=0}^\infty$ is decreasing.
    If it were not, there would exist some $n_1, n_2$ with $n_1 \leq n_2$ and
    $$\sup (a_n)_{n=n_1}^\infty < \sup (a_n)_{n=n_2}^\infty$$
    Then the difference $\delta := \sup (a_n)_{n=n_2}^\infty - \sup (a_n)_{n=n_1}^\infty$ would be positive, so by the properties of the supremum, there exists some $m \geq n_2$ such that
    $$a_m > \sup (a_n)_{n=n_2}^\infty - \frac{\delta}{2} > \sup (a_n)_{n=n_1}^\infty$$
    But this is a contradiction, since $m \geq n_2 \geq n_1$ means we should have $a_m \leq \sup (a_n)_{n=n_1}^\infty$.
    Thus we can conclude that the sequence $(a_n^+)_{n=0}^\infty$ is decreasing.
    
    Next, use the properties of the infimum to take some $M$ such that
    $$L \leq a_M^+ < L - \frac{\epsilon}{2}$$
    Let $m := \max(N, M)$. Since $n \geq M$ and $(a_m^+)_{n=0}^\infty$ is decreasing, we have that
    $$L \leq a_m^+ \leq a_M^+ < L - \frac{\epsilon}{2}$$
    Rewriting, we see that $|a_m^+ - L| < \frac{\epsilon}{2}$.
    Next, apply the definition of $a_m^+$ to get some $n \geq m$ such that
    $$a_m^+ - \frac{\epsilon}{2} < a_n \leq a_m^+$$
    Rewriting once again, we now have some $n \geq m \geq N$ such that $|a_n - a_m^+| < \frac{\epsilon}{2}$. Combining this with the inequality from above, we find
    $$|a_n - L| \leq |a_n - a_m^+| + |a_m^+ - L| < \frac{\epsilon}{2} + \frac{\epsilon}{2} = \epsilon$$
    Since we've found some $n \geq N$ with $|a_n - L| < \epsilon$, $L$ is a limit point of $(a_n)_{n=0}^\infty$.
\end{proof}

The above lemma is perhaps unsurprising, but I wanted to include it for the sake of completeness. Next, another unsurprising but useful lemma.
\begin{lemma}\label{comparison}
    For any sequences $(a_n)_{n=m}^\infty, (b_n)_{n=m}^\infty$ satisfying $a_n \leq b_n$ for all $n$, we have
    \begin{equation}
        \sup (a_n)_{n=m}^\infty \leq \sup (b_n)_{n=m}^\infty
    \end{equation}
    \begin{equation}
        \inf (a_n)_{n=m}^\infty \leq \inf (b_n)_{n=m}^\infty
    \end{equation}
    \begin{equation}
        \limsup_{n \rightarrow \infty} a_n \leq \limsup_{n \rightarrow \infty} b_n
    \end{equation}
\end{lemma}

\begin{proof}
    To prove inequality (1), write $S_b := \sup (b_n)_{n=m}^\infty$ and $S_a := \sup (a_n)_{n=m}^\infty$.
    Assume for contradiction that $S_a > S_b$, and let $\epsilon = S_a - S_b > 0$.
    By the definition of the supremum, there exists some $n$ such that $a_n > S_a - \epsilon = S_b$. But since $S_b$ is the supremum of $(b_n)_{n=0}^\infty$, we have
    that in particular $S_b \geq b_n$. Combining these inequalities, we see that
    $$a_n > S_a - \epsilon = S_b \geq b_n,$$
    which is a contradiction since $a_n \leq b_n$ for all $n$ by assumption.

    To prove inequality (2), write the infima as $I_a$ and $I_b$, and assume that $\epsilon := I_a - I_b > 0$.
    Similar to above, take some $n$ such that $b_n < I_b + \epsilon = I_a$. But $I_a \geq a_k$ for all $k$, which in particular means
    $$b_n < I_b + \epsilon = I_a \leq a_n,$$
    which is again a contradiction.

    Finally, to prove inequality (3), write the limits as $L_a$ and $L_b$ and apply the previous two inequalities to show $L_a \leq L_b$.
    Since $a_n \leq b_n$, inequality (1) gives us
    $$a_k^+ = \sup (a_n)_{n=k}^\infty \leq \sup (b_n)_{n=k}^\infty = b_k^+$$
    for all $k \geq 0$. Since the sequences $(a_n^+)_{n=0}^\infty$ and $(b_n^+)_{n=0}^\infty$ satisfy $a_k^+ \leq b_k^+,$ we can apply inequality (2) to conclude:
    $$L_a = \inf (a_n^+)_{n=0}^\infty \leq \inf (b_n^+)_{n=0}^\infty = L_b$$
\end{proof}

Now we have all the necessary machinery to prove the Bolzano-Weierstrass theorem. In fact, all the hard work is behind us!
\begin{theorem}[Bolzano-Weierstrass]
    Any bounded sequence $(a_n)_{n=0}^\infty$ has a convergent subsequence.
\end{theorem}

\begin{proof}
    Since $(a_n)_{n=0}^\infty$ is bounded, we can take some $M$ such that $-M < a_n < M$ for all $n$.
    Applying Lemma \ref{comparison} to the above inequalities, we find
    $$-M = \limsup_{n \rightarrow \infty} (-M)_{n=0}^\infty \leq \limsup_{n \rightarrow \infty} (a_n)_{n=0}^\infty \leq \limsup_{n \rightarrow \infty} (M)_{n=0}^\infty = M$$
    Since $L := \limsup_{n \rightarrow \infty} (a_n)_{n=0}^\infty$ satisfies $-M \leq L \leq M$, it must be a real number (i.e. not $+\infty$ or $-\infty$).
    So, we can apply Lemma \ref{limpts} to conclude that $L$ is a limit point of $(a_n)_{n=0}^\infty$.
    Because $L$ is a limit point of $(a_n)_{n=0}^\infty$, Proposition \ref{subseqconvtolimpt} allows us to conclude that there exists some subsequence of $(a_n)_{n=0}^\infty$ that converges to $L$.
\end{proof}


\end{document}