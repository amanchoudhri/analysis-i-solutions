\documentclass[11pt, oneside]{article}
\usepackage[margin=1in]{geometry}
\geometry{letterpaper}
\usepackage{graphicx}

\setlength\parindent{0pt}

\usepackage{mathtools}
\usepackage{enumitem}

\usepackage{amsmath}
\usepackage{amssymb}
\usepackage{amsthm}

\newcommand{\R}{\mathbb{R}}
\newcommand{\C}{\mathbb{C}}

\makeatletter
\renewcommand*\env@matrix[1][*\c@MaxMatrixCols c]{%
  \hskip -\arraycolsep
  \let\@ifnextchar\new@ifnextchar
  \array{#1}}
\makeatother

\newcommand{\ip}[1]{\left \langle #1 \right \rangle}

\newlist{alphalist}{enumerate}{1}
% label must be specified using \setlist
\setlist[alphalist,1]{label=\alph*.}


\title{MATH 4061 - Homework 1}
\author{Aman Choudhri - \texttt{ac4972@columbia.edu}}

\begin{document}

\maketitle
\subsection*{Problem 1}
Show that $\C$ is a field, where we define a complex number to be a pair $(a, b)$ with $a, b \in \R$.
Take $x = (a, b), y = (c, d) \in \C$. We define two binary operations $+, \cdot: \C \rightarrow \C$ as follows:
$$
x + y := (a + c, b + d)
$$
$$
x \cdot y := (ac - bd, ad + bc)
$$
In addition, define $1 \in \C$ by $(1, 0)$ and $0 \in \C$ by $(0, 0)$.
\begin{proof}
    First show that it satisfies the axioms for addition. Commutativity:
    $$x + y = (a + c, b + d) = (c + a, d + b) = y + x,$$
    where the middle step follows because $\R$ is a field. Let $z \in \C$ by $(e, f)$. Associativity:
    $$x + (y + z) = (a + (c + e), b + (d + f)) = ((a + c) + e, (b + d) + f) = (x + y) + z$$
    Show that $x + 0 = x$ for all $x \in \C$:
    $$x + 0 = (a + 0, b + 0) = (a, b) = x$$
    Lastly, define $-x := (-a, -b)$, which we know exists since $\R$ is a field, and show $x + (-x) = 0$:
    $$x + (-x) = (a + (-a), b + (-b)) = (0, 0) = 0$$
    Now the axioms for multiplication. Commutativity:
    $$xy = (ac - bd, ad + bc) = (ca - db, cb + ad) = yx$$
    Associativity:
    \begin{align*}
        x(yz) &= x(ce - df, cf + de) \\
        &= (a(ce - df) - b(cf + de), a(cf + de) + b(ce - df)) \\
        &= (ace - adf - bcf - bde, acf + ade + bce - bdf) \\
        &= (ace - bde - adf - bcf , ade + bce + acf - bdf) \\
        &= (e(ac - bd) - f(ad + bc), e(ad + bc) + f(ac - bd)) \\
        &= (ac - bd, ad + bc) \cdot (e, f) \\
        &= ((a, b) \cdot (c, d)) \cdot z \\
        &= (xy)z
    \end{align*}
    Distributivity:
    \begin{align*}
        x(y + z) = x(c + e, d + f) &= (a(c + e) - b(d + f), a(d+f) + b(c + e)) \\
        &= (ac + ae - bd - bf, ad + af + bc + be) \\
        &= (ac + -bd, ad + bc) + (ae - bf, af + be) \\
        &= xy + xz
    \end{align*}
    Show that $1x = x$ for all $x \in \C$:
    $$1x = (1, 0) \cdot (a, b) = (1a - 0b, 1b + 0a) = (a, b) = x$$
    For some nonzero complex number $x = (a, b)$, define $x^{-1}$ to be
    $$\left(\frac{a}{a^2 + b^2}, \ \frac{-b}{a^2 + b^2} \right)$$
    This is well defined because $x$ is nonzero and so $a$ and $b$ are nonzero as well. Now show that this functions as a multiplicative inverse:
    \begin{align*}
        xx^{-1} &= \left(a \left(\frac{a}{a^2 + b^2} \right) - b\left(\frac{-b}{a^2 + b^2}\right), \ a \left(\frac{-b}{a^2 + b^2}\right) + b \left(\frac{a}{a^2 + b^2}\right)\right) \\
        &= \left( \frac{a^2}{a^2 + b^2} + \frac{b^2}{a^2 + b^2}, \ \frac{-ab}{a^2 + b^2} + \frac{ab}{a^2 + b^2} \right) \\
        &= \left( \frac{a^2 + b^2}{a^2 + b^2}, \ 0 \right) \\
        &= \left( 1, 0 \right) \\
        &= 1
    \end{align*}
And we're done!
\end{proof}
Next, prove a few relations for $z, w \in \C$. Write $z = (a, b), w = (c, d)$.
\begin{enumerate}
    \item $\overline{z + w} = \overline{(a + c, b + d)} = (a + c, -b - d) = (a, - b) + (c, -d) = \overline{z} + \overline{w}$.
    \item $\overline{zw} = \overline{(ac - bd, ad + bc)} = (ac - bd, -ad - bc) = (a, -b) \cdot (c, -d) = \overline{z} \cdot \overline{w}$.
    \item $z + \overline{z} = (a, b) + (a, - b) = (2a, 0) = 2 \operatorname{Re}(z)$.\\
    $z - \overline{z} = (a, b) - (a, -b) = (0, 2b) = 2b (0, 1) = 2i \operatorname{Im}(z)$
    \item $|z| = z\overline{z} = (a, b) \cdot (a, -b) = (a^2 + b^2, -ab + ab) = (a^2 + b^2, 0)$. Since $a^2$ and $b^2$ are both nonnegative, their sum is nonnegative and thus $|z| \geq 0$ for all $z \in \C$.
    
    $|z| = 0 \iff z\overline{z} = 0 \iff (a^2 + b^2, 0) = 0 \iff a^2 + b^2 = 0 \iff a = 0 \wedge b = 0 \iff z = 0$
    \item To show $|\overline{z}| = |z|$, first note that $\overline{\overline{z}} = z$ since $\overline{\overline{(a, b)}} = \overline{(a, -b)} = (a, b)$.
    With that, it's clear that $|\overline{z}| = \overline{z} \cdot \overline{\overline{z}} = \overline{z} z = |z|$.
\end{enumerate}

\section*{Problem 2}
Now, prove a series of facts about $\R^n$. Take $x, y, z \in \R^n$ and $\alpha \in \R$.
\begin{enumerate}
    \item Linearity of the inner product.
        $$\ip{x + y, z} = \sum_{i = 1}^n (x_i + y_i) (z_i) = \sum_{i = 1}^n (x_i z_i) + (y_i z_i) = \sum_{i = 1}^n (x_i z_i) + \sum_{i = 1}^n (y_i z_i) = \ip{x, z} + \ip{y, z}$$
        $$\ip{\alpha x, y} = \sum_{i=1}^n \alpha x_i y_i = \alpha \sum_{i=1}^n x_i y_i = \alpha \ip{x, y}$$
    \item Symmetry of the real inner product.
    $$\ip{x, y} = \sum_{i=1}^n x_i y_i = \sum_{i=1}^n y_i x_i = \ip{y, x}$$
    \item Positive definiteness.
    $$\ip{x, x} = \sum_{i=1}^n x_i^2,$$
    which is nonnegative since each term $x_i^2 \geq 0$. In addition, we have
    $$\ip{x, x} = 0 \iff \sum_{i=1}^n x_i^2 = 0 \iff (\forall i \leq n) \ x_i = 0  \iff x = 0$$
    \item Cauchy-Schwarz.
    $$|\ip{x, y}| = \left| \sum_{i=1}^n x_i y_i \right| \leq \sum_{i=1}^n |x_iy_i| \leq \sum_{i=1}^n |x_i||y_i| \leq \sqrt{\left( \sum_{i=1}^n x_i^2 \right) \left( \sum_{i=1}^n y_i^2 \right)} = \|x\| \|y\|,$$
    where the pentultimate step is given by the Schwarz inequality.
\end{enumerate}
\end{document}